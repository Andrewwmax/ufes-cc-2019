%%%%%%%%%%%%%%%%%%%%%%%%%%%%%%%%%%%%%%%%%%%%%%%%%%%%%%%%%%%%%%%
%
% Welcome to Overleaf --- just edit your LaTeX on the left,
% and we'll compile it for you on the right. If you give
% someone the link to this page, they can edit at the same
% time. See the help menu above for more info. Enjoy!
%
% Note: you can export the pdf to see the result at full
% resolution.
%
%%%%%%%%%%%%%%%%%%%%%%%%%%%%%%%%%%%%%%%%%%%%%%%%%%%%%%%%%%%%%%%
\documentclass{beamer}

\usepackage{tikz}

\usepackage{verbatim}
\usetikzlibrary{arrows,shapes}


\begin{document}
\begin{comment}
:Title: Prim's algorithm
:Tags: Beamer, Layers, Foreach, Graphs
:Use page: 11


A step by step example of the `Prim's algorithm`_ for finding the `minimum
spanning tree`_. Animated using Beamer 
overlays.

.. _Prim's algorithm: http://en.wikipedia.org/wiki/Prim%27s_algorithm
.. _Minimum spanning tree: http://en.wikipedia.org/wiki/Minimum_spanning_tree

| Source: Adapted from an example on Wikipedia_

.. _Wikipedia: http://en.wikipedia.org/wiki/Prim%27s_algorithm
\end{comment}

% Declare layers
\pgfdeclarelayer{background}
\pgfsetlayers{background,main}

\begin{frame}
\frametitle{Dijkstra}

%% Adjacency matrix of graph
%% \  a  b  c  d  e  f  g
%% a  x  7     5
%% b  7  x  8  9  7
%% c     8  x     5
%% d  5  9     x 15  6
%% e     7  5 15  x  8  9
%% f           6  8  x 11
%% g              9  11 x

\tikzstyle{vertex}=[circle,fill=black!25,minimum size=20pt,inner sep=0pt]
\tikzstyle{selected vertex} = [vertex, fill=green!24]
\tikzstyle{next vertex} = [vertex, fill=red!24]
\tikzstyle{edge} = [draw,thick,-]
\tikzstyle{weight} = [font=\small]
\tikzstyle{selected edge} = [draw,line width=5pt,-,red!50]
\tikzstyle{path edge} = [draw,line width=5pt,-,green!50]
\tikzstyle{ignored edge} = [draw,line width=5pt,-,black!20]

\begin{figure}
\begin{tikzpicture}[scale=1, auto,swap]
    % First we draw the vertices
    \foreach \pos/\name in {{(2,2)/1}, {(2,-2)/7}, {(4,2)/2},{(0,0)/0}, {(8,0)/4}, {(4,0)/8},{(6,2)/3}, {(4,-2)/6}, {(6,-2)/5}}
        \node[vertex] (\name) at \pos {$\name$};
        
    % Connect vertices with edges and draw weights
    \foreach \source/ \dest /\weight in { 0/1/4, 1/2/8, 2/3/7, 3/4/9,4/5/10, 5/6/2, 6/7/1, 7/8/7,0/7/8, 0/1/4, 1/7/11, 2/8/2, 8/6/6, 2/5/4, 3/5/14}
        \path[edge] (\source) -- node[weight] {$\weight$} (\dest);
    
    % Start animating the vertex and edge selection. 
    \foreach \vertex / \fr in { 1/1,7/1,2/2,2/3,8/3,6/3,5/4,4/5,3/5}
        \path<\fr-> node[next vertex] at (\vertex) {$\vertex$};

    \foreach \vertex / \fr in { 0/1,1/2,7/3,6/4,5/5,2/6,8/7,3/8,4/9}
        \path<\fr-> node[selected vertex] at (\vertex) {$\vertex$};
    
    % For convenience we use a background layer to highlight edges
    % This way we don't have to worry about the highlighting covering
    % weight labels. 
    \begin{pgfonlayer}{background}
        \pause
        \foreach \source / \dest / \fr in { 0/7/2, 1/7/2, 1/2/2, 
                                            1/2/3, 7/8/3, 7/6/3, 
                                            1/2/4, 6/8/4, 6/5/4, 
                                            1/2/5, 5/2/5, 5/3/5, 5/4/5, 6/8/5, 6/5/5,
                                            2/3/6, 2/8/6, 5/2/6, 5/3/6, 5/4/6, 6/8/6,
                                            2/3/7, 5/3/7, 5/4/7,
                                            5/4/8, 3/4/8, 3/5/8}
            \path<\fr>[selected edge] (\source.center) -- (\dest.center);

        \foreach \source / \dest / \fr in {0/1/1, 0/7/3, 7/6/4, 1/2/5, 6/5/6, 2/8/7, 2/3/8, 5/4/9}
            \path<\fr->[path edge] (\source.center) -- (\dest.center);
            
        \foreach \source / \dest / \fr in {1/7/3, 7/8/4, 2/5/6, 6/8/6, 3/5/8, 3/4/9}
            \path<\fr->[ignored edge] (\source.center) -- (\dest.center);
            
    \end{pgfonlayer}
\end{tikzpicture}
\end{figure}

\end{frame}

\end{document}